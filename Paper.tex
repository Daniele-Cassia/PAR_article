\documentclass[sigconf]{webmedia}
%webmeda.cls is adapted from acmart.cls for use in Proceedings of the Brazilian Symposium on Multimedia and the Web (WebMedia)

%%
%% \BibTeX command to typeset the BibTeX logo in the docs
\AtBeginDocument{%
  \providecommand\BibTeX{{%
    \normalfont B\kern-0.5em{\scshape i\kern-0.25em b}\kern-0.8em\TeX}}}

\settopmatter{printacmref=false, printfolios=false}

%% Choose the appropriate event.
\setevent{webmedia}
\proceedingsDetails[WebMedia’2024]{Proceedings of the Brazilian Symposium on Multimedia and the Web}{2024}{Juiz de Fora, Brazil}
\ISSN{XXXX-XXXX}

%\setevent{ctd}
%\proceedingsDetails[CTD’2024]{VI Concurso de Teses e Disserta{\c c}{\~o}es (CTD 2024). Anais Estendidos do XXX Simpósio Brasileiro de Sistemas Multimídia e Web}{2024}{Juiz de Fora/MG, Brazil}
%\ISSN{2596-1683}

%\setevent{ctic}
%\proceedingsDetails[CTIC’2024]{IV Concurso de Trabalhos de Inicia{\c c}{\~a}o Cient{\'i}fica (CTIC 2024). Anais Estendidos do XXX Simpósio Brasileiro de Sistemas Multimídia e Web}{2024}{Juiz de Fora/MG, Brazil}
%\ISSN{2596-1683}

%\setevent{wfa}
%\proceedingsDetails[WFA’2024]{XXII Workshop de Ferramentas e Aplica{\c c}{\~o}es (WFA 2024) (CTIC 2024). Anais Estendidos do XXX Simpósio Brasileiro de Sistemas Multimídia e Web}{2024}{Juiz de Fora/MG, Brazil}
%\ISSN{2596-1683}

%% end of the preamble, the start of the body of the document source.
\begin{document}

%%
%% The "title" command has an optional parameter,
%% allowing the author to define a "short title" to be used in page headers.
\title{PAR Digital: Tecnologia em Prol do Ensino Inclusivo}
% \title{PAR Digital: Tecnologia em Prol da Educação Inclusiva}

% \subtitle{Subtitle (optional)}
%%
%% The "author" command and its associated commands are used to define
%% the authors and their affiliations.
%% Of note is the shared affiliation of the first two authors and the
%% "authornote" and "authornotemark" commands
%% used to denote shared contribution to the research.


\author{Daniele Cássia}
\affiliation{%
  \institution{Departamento de Ciência de Computação}
  \institution{Universidade Federal de Minas Gerais}
  \city{Belo Horizonte}
  \country{Brasil}}
\email{danielecassia@ufmg.br}

\vspace{1cm}





%%
%% By default, the full list of authors will be used in the page
%% headers. Often, this list is too long, and will overlap
%% other information printed in the page headers. This command allows
%% the author to define a more concise list
%% of authors' names for this purpose.
% \renewcommand{\shortauthors}{Trovato et al.}

%%
%% The abstract is a short summary of the work to be presented in the
%% article.

\begin{abstract}
  Este artigo apresenta uma descrição dos objetivos e funcionalidades do
  PAR Digital, demonstrando como o software pode transformar a prática
  educacional inclusiva.
\end{abstract}


%%
%% Keywords. The author(s) should pick words that accurately describe
%% the work being presented. Separate the keywords with commas.
\keywords{software, tecnologia, educação, plano educacional individualizado, inclusão}

%% A "teaser" image appears between the author and affiliation
%% information and the body of the document, and typically spans the
%% page.
% \begin{teaserfigure}
%   \includegraphics[width=\textwidth]{sampleteaser}
%   \caption{Seattle Mariners at Spring Training, 2010.}
%   \Description{Enjoying the baseball game from the third-base
%   seats. Ichiro Suzuki preparing to bat.}
%   \label{fig:teaser}
% \end{teaserfigure}

%%
%% This command processes the author and affiliation and title
%% information and builds the first part of the formatted document.
\maketitle

\section{Introdução}

O PAR Digital é uma ferramenta tecnológica desenvolvida com base em anos de estudo
e pesquisa pela Faculdade de Educação da Universidade Federal de Minas Gerais, PAR
significa Planejar, Aplicar, Rever, ações necessárias para que o ensino seja eficaz
para os alunos com deficiência.
\vspace{0.5cm}

\includegraphics[scale=0.12]{./imgs/capa}

\section{Objetivo}
O objetivo do PAR Digital é proporcionar uma ferramenta acessível e intuitiva
que facilite o preenchimento e a gestão do Plano Educacional Individualizado (PEI),
essencial para o desenvolvimento educacional dos alunos com deficiência. Desenvolvido
a partir de princípios do Desenho Universal de Aprendizagem (DUA) e de uma parceria
com o Atendimento Educacional Especializado (AEE), o sistema visa integrar-se de
 maneira eficiente na rotina diária dos educadores, promovendo um ensino inclusivo
 e personalizado.
 % \url{https://www.acm.org/publications/proceedings-template}.
 
 \section{Arquitetura}
 
 A estrutura do projeto é organizada de maneira a facilitar 
 a manutenção e a expansão do mesmo.
 
 As principais tecnologias utilizadas no projeto incluem React.js, uma biblioteca
 JavaScript para construção de interfaces de usuário, escolhida por sua eficiência
na criação de componentes reutilizáveis e desempenho otimizado. O TypeScript é
usado para adicionar tipagem estática ao código JavaScript, aumentando a robustez
e facilitando a manutenção. O Redux é empregado para o gerenciamento de estado da
aplicação, ideal para aplicações de médio e grande porte que necessitam de um
controle mais sofisticado do estado. A biblioteca Material-UI é utilizada para
implementar o design system do Google Material Design, enquanto o Axios serve
como cliente HTTP para realizar requisições a APIs. Ferramentas como Jest e
React Testing Library são utilizadas para testes automatizados, garantindo a
qualidade e a funcionalidade do código.

O fluxo de dados na aplicação é gerenciado pelo Redux, que centraliza o estado da
aplicação em um único store. As ações são despachadas a partir dos componentes,
que são tratadas pelos `reducers` para atualizar o estado global. Essa abordagem
facilita a depuração e o desenvolvimento de novas funcionalidades, uma vez que o
estado da aplicação se torna previsível e controlado.

A estilização da aplicação é feita utilizando CSS-in-JS com a biblioteca Material-UI,
 permitindo uma aplicação consistente do design system através dos componentes.
 A utilização de temas facilita a customização e a manutenção do estilo visual da
 aplicação.

Para a integração e entrega contínua (CI/CD), o projeto utiliza
o GitHub Actions, configurado para executar testes automatizados e builds a cada
commit, garantindo que a aplicação se mantenha estável e pronta para deployment.
O deploy é realizado em um servidor localizado na UFMG, permitindo uma escalabilidade
 fácil e eficiente.
 \vspace{0.2cm}

 \includegraphics[scale=0.17]{./imgs/arquitetura.png}

\section{Funcionalidades}

O projeto oferece diversas funcionalidades voltadas para o suporte
educacional de estudantes com necessidades especiais. A seguir, são
descritas as principais funcionalidades do sistema, cada uma desempenhando
 um papel crucial na promoção de um ambiente inclusivo e eficaz para o
 aprendizado.
\vspace{0.5cm}

\includegraphics[scale=0.12]{./imgs/painel}

\subsection{CRUD de Usuários}
No sistema, os {\bfseries administradores} são responsáveis por gerenciar as
escolas, visualizando relatórios e detalhes como professores e
estratégias.



As {\bfseries escolas}  criam professores, responsáveis e
desenvolvem o PEI, além de visualizar estratégias e dados dos alunos
e professores.

Os {\bfseries coordenadores}  acompanham o progresso dos alunos
e estratégias, visualizando o andamento dos alunos.

Os {\bfseries professores
regulares}  são os usuários chave, criando e aplicando estratégias, e
adicionando relatórios de progresso.

Os {\bfseries professores AEE}  auxiliam
os regulares, visualizando estratégias do PEI e relatórios, e
participando de fóruns.

Os {\bfseries responsáveis}  adicionam documentos
necessários e acompanham o desenvolvimento dos alunos, visualizando
seus dados e relatórios parciais.


\subsection{Criação do PEI}
A funcionalidade de criação do Plano Educacional Individualizado
(PEI) é uma das mais importantes, permitindo que educadores e 
coordenadores elaborem planos personalizados para atender às 
necessidades específicas de cada aluno.
\vspace{0.5cm}

\includegraphics[scale=0.12]{./imgs/pei}


\subsection{Estratégias do PEI}
Dentro do PEI, os educadores podem definir e documentar diversas 
estratégias pedagógicas adaptadas às necessidades do aluno. Essas 
estratégias são orientações práticas que guiam os professores na 
implementação do plano e ajudam a monitorar o progresso do aluno.
\vspace{0.5cm}

\includegraphics[scale=0.12]{./imgs/estrategia}

\subsection{Relatório sobre Estratégias}
A funcionalidade de relatórios permite que educadores e coordenadores 
acompanhem o andamento das estratégias definidas no PEI. 
Esses relatórios são fundamentais para avaliar o progresso do aluno, 
identificar áreas que necessitam de ajustes e garantir que as 
estratégias estão sendo eficazmente implementadas.
\vspace{0.5cm}

\includegraphics[scale=0.12]{./imgs/relatorio}


\subsection{Fórum de Acompanhamento do Aluno}
O fórum de acompanhamento do aluno é uma plataforma colaborativa onde 
professores, coordenadores e outros profissionais podem discutir o 
progresso do aluno, compartilhar insights e propor ajustes no PEI. 
Este fórum promove uma abordagem colaborativa e integrada ao 
acompanhamento educacional.
\vspace{0.5cm}

\includegraphics[scale=0.12]{./imgs/forum}

\subsection{Reutilização de Estratégias}
O sistema possui um banco de estratégias pedagógicas que podem ser 
reutilizadas. Educadores podem consultar esse banco para encontrar 
estratégias que já foram aplicadas com sucesso em situações similares
, facilitando a implementação de práticas comprovadamente eficazes.
\vspace{0.5cm}

\includegraphics[scale=0.12]{./imgs/banco}

\subsection{Compartilhamento de Recursos}
Uma funcionalidade vital do PAR Digital é o compartilhamento de 
recursos, especialmente Tecnologias Assistivas. Esse recurso 
permite que professores e coordenadores acessem ferramentas e 
materiais que podem ser utilizados para apoiar o aprendizado 
dos alunos com necessidades especiais.
\vspace{0.5cm}

\includegraphics[scale=0.12]{./imgs/materialAdd}

\subsection{Perfil do Aluno}
O sistema mantém um perfil detalhado do aluno, que inclui informações
 educacionais e fundamentais. Esse perfil permite que educadores e 
 coordenadores tenham uma visão abrangente das necessidades, 
 capacidades e progressos do aluno, facilitando a personalização 
 do ensino e a monitorização do desenvolvimento educacional.
 \vspace{0.5cm}

\includegraphics[scale=0.12]{./imgs/perfil}

\section{Licença}
Software acadêmico, licença de tecnologias livres que está
 sendo utilizado de forma de cunho social?


\section{Perspectivas}

\subsection{Perspectiva Acadêmica}
O PAR Digital representa uma inovação significativa no campo 
da educação inclusiva, oferecendo uma ferramenta tecnológica 
poderosa para a gestão eficiente de Plano Educacional 
Individualizado (PEI). Academicamente, essa ferramenta pode 
servir como um catalisador para a pesquisa e o 
desenvolvimento contínuo de estratégias pedagógicas 
adaptativas. 

O acesso a dados detalhados sobre o progresso dos alunos, 
bem como o compartilhamento de estratégias bem-sucedidas 
entre os educadores, pode promover a produção de conhecimento
 acadêmico relevante na área da educação inclusiva. 

Além disso, o projeto pode facilitar estudos longitudinais 
sobre o impacto de intervenções específicas no 
desenvolvimento educacional de alunos com deficiência, 
fornecendo insights valiosos para a comunidade acadêmica.

\subsection{Perspectiva Social}
Socialmente, o PAR Digital tem o potencial de promover a 
inclusão e a igualdade de oportunidades na educação. Ao 
oferecer uma plataforma acessível e intuitiva para a 
gestão de PEIs, o sistema capacita os educadores a 
oferecer um suporte mais eficaz aos alunos com deficiência, 
adaptando seus métodos de ensino às necessidades 
individuais de cada estudante. Isso não apenas melhora 
a experiência educacional desses alunos, mas também 
fortalece a coesão social ao reconhecer e valorizar a 
diversidade no ambiente escolar.

Além disso, ao envolver os pais e responsáveis no processo 
educacional por meio do acompanhamento do progresso dos 
alunos, o software promove uma parceria colaborativa entre 
escola e comunidade, fortalecendo os laços sociais e a 
confiança na instituição educacional.



\section{Conclusão e Trabalhos Futuros}
TODO:Fazer fechamento do artigo e direções de funcionalidades futuras

\end{document}
\endinput
%%
%% End of file `sample-sigconf.tex'.
